\documentclass[a4paper,12pt]{article}
\usepackage{graphicx}
\usepackage{float}
\usepackage[english,russian]{babel}

\title{1.1.4 Измерение интенсивности радиационного фона}
\author{Тимур Байдюсенов Б01-302}
\date{15.09.2023}

\begin{document}
\maketitle

\section{Аннотация}
В работе измеряется интенсивность радиационного фона. Данные фиксируются с помощью счётчика Гейгера-Мюллера(CTC-6). Исследуются ошибки результатов.

\section{Теоретические сведения}
Количество отсчётов в одном опыте подчиняется распределению Пуассона, т.к. регистрация частиц однородна по времени и каждая следующая не зависит от предыдущего.
Стандартная ошибка отдельного измерения находится по формуле:
\begin{equation}
\sigma=\sqrt{n}
\end{equation}
Значит результат измерений записывается так: 
\begin{equation}
n_0=n\pm\sqrt{n}
\end{equation}
При $N$ измерениях среднее значение числа сосчитанных за одно измерение частиц равно:
\begin{equation}
\overline{n}=\frac{1}{N}\sum_{i=1}^{N} {n_i}
\end{equation} 
Стандартную ошибку отдельного измерения можно оценить по формуле: 
\begin{equation}
\sigma_{\mbox{отд}}=\sqrt{\frac{1}{N}\sum_{i=1}^{N}{(n_i-\overline{n})^2}}
\end{equation}
Ближе всего к значению $ \sigma_{\mbox{отд}} $ лежит величина $ \sqrt{\overline{n}} $, тогда:
\begin{equation}
\sigma_{\mbox{отд}} \approx \sqrt{{\overline{n}}}
\end{equation}
Как показывает теория вероятностей стандартная ошибка отклонения $\overline{n}$ от $n_0$ может быть определена так:
\begin{equation}
\sigma_{\overline{n}}=\frac{1}{N}\sqrt{\sum_{i=1}^N{(n_i-\overline{n})^2}}=\frac{\sigma_{\mbox{отд}}}{\sqrt{N}}
\end{equation}
Относительная ошибка отдельного измерения:
\begin{equation}
\varepsilon_{\mbox{отд}}=\frac{\sigma_{\mbox{отд}}}{n_i} \approx \frac{1}{\sqrt{n_i}}
\end{equation}
Аналогичным образом определяется относительная ошибка в определении среднего по всем измерениям значения $\overline{n}$:
\begin{equation}
\varepsilon_{\overline{n}}=\frac{\sigma_{\overline{n}}}{\overline{n}}=\frac{\sigma_{\mbox{отд}}}{\overline{n}\sqrt{N}}\approx\frac{1}{\sqrt{\overline{n}N}}
\end{equation}

\section{Оборудование}
\begin{figure}[H]
\centering
\includegraphics[width=0.25\textwidth]{счётчик}
\caption{Схема включения счетчика} 
\end{figure}

Обнаружить космичекие лучи можно с помощью ионизации, которую они производят, используя счетчик Гейгера-Мюллера. Счетчик представляет собой наполненный газом сосуд с двумя электродами: металлическим цилиндром и нитью. Частицы космических лучей ионизируют газ, выбивая электроны из стенок сосуда и создавая лавину электронов: сталкиваясь с молекулами газа, выбивают из них электроны. Так, получается лавина электронов, в следствии чего, через счетчик увеличивается ток и регистрируется частица. 

\section{Результаты измерений и обработка данных}

\begin{table}[H]
\centering
\caption{Число срабатываний счетчика за 20 с}
\begin{tabular}{|c|c|c|c|c|c|c|c|c|c|c|}
\hline 
№ опыта : & 1 & 2 & 3 & 4 & 5 & 6 & 7 & 8 & 9 & 10 \\ 
\hline 
0 : & 0 & 8 & 24 & 30 & 33 & 25 & 36 & 26 & 33 & 30 \\ 
10 : & 27 & 24 & 23 & 31 & 24 & 23 & 31 & 24 & 30 & 35 \\ 
20 : & 25 & 29 & 28 & 25 & 19 & 36 & 16 & 23 & 21 & 23 \\ 
30 : & 30 & 14 & 31 & 25 & 25 & 33 & 36 & 32 & 23 & 27 \\ 
40 : & 23 & 21 & 19 & 16 & 21 & 22 & 27 & 23 & 20 & 34 \\ 
50 : & 35 & 21 & 16 & 26 & 28 & 29 & 32 & 28 & 30 & 27 \\ 
60 : & 31 & 25 & 21 & 31 & 19 & 26 & 25 & 27 & 17 & 31 \\
70 : & 27 & 25 & 22 & 24 & 23 & 16 & 25 & 29 & 27 & 32 \\ 
80 : & 23 & 23 & 22 & 28 & 29 & 28 & 26 & 29 & 35 & 20 \\ 
90 : & 25 & 29 & 19 & 23 & 21 & 26 & 19 & 24 & 22 & 32 \\ 
100 : & 25 & 33 & 40 & 31 & 28 & 30 & 27 & 33 & 32 & 27 \\ 
110 : & 31 & 23 & 25 & 31 & 30 & 37 & 33 & 32 & 33 & 21 \\ 
120 : & 21 & 33 & 29 & 31 & 23 & 29 & 30 & 27 & 31 & 21 \\ 
130 : & 19 & 29 & 20 & 28 & 40 & 20 & 25 & 29 & 31 & 32 \\ 
140 : & 29 & 15 & 24 & 31 & 28 & 26 & 36 & 24 & 20 & 31 \\ 
150 : & 25 & 20 & 22 & 32 & 25 & 34 & 32 & 33 & 28 & 33 \\ 
160 : & 29 & 25 & 20 & 25 & 17 & 31 & 33 & 21 & 33 & 27 \\ 
170 : & 26 & 25 & 31 & 34 & 26 & 25 & 31 & 16 & 21 & 26 \\ 
180 : & 32 & 26 & 27 & 33 & 32 & 26 & 22 & 25 & 34 & 19 \\ 
190 : & 33 & 27 & 31 & 27 & 31 & 27 & 25 & 25 & 22 & 35 \\ 
\hline 
\end{tabular} 
\end{table}

\begin{table}[H]
\centering
\caption{Данные для построения гистограммы распределения числа срабатываний счетчика за 10 с} \label{10c}
\begin{tabular}{|c|c|c|c|c|c|}
\hline 
Число импульсов $n_i$ & 4 & 5 & 6 & 7 & 8 \\ 
\hline 
Число случаев & 2 & 6 & 6 & 6 & 22 \\ 
\hline 
Доля случаев $w_n$ & 0.005 & 0.015 & 0.015 & 0.015 & 0.055 \\ 
\hline 
\hline 
Число импульсов $n_i$ & 9 & 10 & 11 & 12 & 13 \\ 
\hline 
Число случаев & 22 & 30 & 44 & 37 & 47 \\ 
\hline 
Доля случаев $w_n$ & 0.055 & 0.075 & 0.11 & 0.0925 & 0.1175 \\ 
\hline 
\hline 
Число импульсов $n_i$ & 14 & 15 & 16 & 17 & 18 \\ 
\hline 
Число случаев & 54 & 30 & 22 & 25 & 19 \\ 
\hline 
Доля случаев $w_n$ & 0.135 & 0.075 & 0.055 & 0.0625 & 0.0475 \\ 
\hline 
\hline 
Число импульсов $n_i$ & 19 & 20 & 21 & 22 & 23 \\ 
\hline 
Число случаев & 8 & 7 & 5 & 2 & 1 \\ 
\hline 
Доля случаев $w_n$ & 0.02 & 0.0175 & 0.0125 & 0.005 & 0.0025 \\ 
\hline 
\hline 
Число импульсов $n_i$ & 24 & 25 & 26 & 27 & 28\\ 
\hline 
Число случаев & 0 & 1 & 0 & 0 & 1\\ 
\hline 
Доля случаев $w_n$ & 0 & 0.0025 & 0 & 0 & 0.0025\\ 
\hline 

\end{tabular} 
\end{table}

\begin{table}[H]
\centering
\caption{Данные для построения гистограммы распределения числа срабатываний счетчика за 40 с} \label{40c}
\begin{tabular}{|c|c|c|c|c|c|}
\hline 
Число импульсов $n_i$ & 42 & 43 & 44 & 45 & 46\\ 
\hline 
Число случаев & 2 & 1 & 6 & 2 & 4 \\ 
\hline 
Доля случаев $w_n$ & 0.02 & 0.01 & 0.06 & 0.02 & 0.04 \\ 
\hline 
\hline 
Число импульсов $n_i$ & 47 & 48 & 49 & 50 & 51\\ 
\hline 
Число случаев & 4 & 4 & 5 & 7 & 6 \\ 
\hline 
Доля случаев $w_n$ & 0.04 & 0.04 & 0.05 & 0.06 & 0.06 \\ 
\hline 
\hline 
Число импульсов $n_i$ & 52 & 53 & 54 & 55 & 56\\ 
\hline 
Число случаев & 4 & 6 & 10 & 6 & 2 \\ 
\hline 
Доля случаев $w_n$ & 0.04 & 0.06 & 0.1 & 0.06 & 0.02 \\ 
\hline 
\hline
Число импульсов $n_i$ & 57 & 58 & 59 & 60 & 61\\ 
\hline 
Число случаев & 4 & 7 & 2 & 1 & 3 \\ 
\hline 
Доля случаев $w_n$ & 0.04 & 0.07 & 0.02 & 0.01 & 0.03 \\ 
 
\hline 

\end{tabular} 
\end{table}

\end{document}